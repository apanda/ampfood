\section{Discussion}
Much of our accuracy can be attributed to environmental features. In particular the table is in a fixed position, all
food on the table is acquired from a few restaurants and somehow people tend to place boxes in the same position on the
table. As a result the problem is highly constrained and provided sufficient input data the SVM can achieve high
accuracy.

However there is the issue of why HOG vectors do not work. Based on the images in
Figure~\ref{fig:food-hog},~\ref{fig:other-hog} it appears that HOG does not produce sufficiently different feature
vectors for cases where the table contains food or any other object. Part of the problem is that our camera position
limits the amount of depth information that can be captured and hence both boxes and flat objects look roughly
identical.

While using just edges from the whole image works well for the binary classification problem, we would eventually like
to be able to classify both the kind of food and the amount which might require other features.


\section{Future Work}
In future work, we plan to add detection of the rate at which the food is being consumed.
This would allow us to email users with an extra alert when food is being consumed quickly:
``come quickly, the food won't be around for long!''.  We also would like to make the detection
more sophisticated in the event that more food is added to the table.  Currently, we 
only inform users when the table's state changes from not having food to having food.
However, there are sometimes cases when bad food lingers on the table for a while, and in the
meantime, new food is added.  We'd like to improve the algorithm to send alerts in this case
as well. One final area of future work is detecting the type of food on the table. Lunches
from Gregoire are, for example, far more appealing to most graduate students than leftover salad,
and it would be useful if this information were included in emails.
