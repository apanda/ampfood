\section{Introduction}
The importance of free food in the life of a graduate student is a common motif in popular culture. In the real world,
students in the $4^{th}$ floor of Soda Hall spend a small portion of their afternoon polling the kitchen for lunch. In
this project we attempted to use computer vision to minimize research productivity lost when polling the kitchen.
Broadly our idea was to install a camera in the RADLab kitchen (feed available at
\url{http://radlab.cs.berkeley.edu/foodcam}) and use the images acquired to train a SVM that could label pictures as
either containing food or not containing food. 

The RADLab kitchen did not previously provide any video feed and did not provide an easy way to access a stream of
images. As a result we had the freedom to design both the image acquisition and recognition pipeline for this project.
As we show below carefully selecting the camera viewpoint greatly improved our recognition accuracy. Furthermore
selecting the camera viewpoint is practical in problems similar to ours (where the aim is to recognize the presence of
an item on a fixed target). 

We next describe different parts of our processing pipeline.
