\section{Failure Modes}
The foodcam suffers from a number of failure modes.  First of all, there are a variety of objects
that look similar to food. Many food containers are square shaped, making them similar to
papers that have been placed on the table.  A second failure mode is when someone is sitting
at the table and eating food.  In this case, the problem is more difficult to fix: the FoodCam
is answering the ``Is there food on the table?'' question correctly, but the food is not
available to be consumed by others. Finally, some students amused by the FoodCam attempted
to trick it with fake food.  For example, as shown in Figure~\ref{fig:??}, one witty student
placed a drawing of a drumstick on the table. We anticipate that amusement at the FoodCam
and corresponding pranks will die out over time, but in the meantime, the FoodCam sometimes
falsely identifies these pranks as actual food.

\section{Future Work}
In future work, we plan to add detection of the rate at which the food is being consumed.
This would allow us to email users with an extra alert when food is being consumed quickly:
``come quickly, the food won't be around for long!''.  We also would like to make the detection
more sophisticated in the event that more food is added to the table.  Currently, the
FoodCam only emails users when the table's state changes from not having food to having food.
However, there are sometimes cases when bad food lingers on the table for a while, and in the
meantime, new food is added.  We'd like to improve the algorithm to send alerts in this case
as well. One final area of future work is detecting the type of food on the table. Lunches
from Gregoire are, for example, far more appealing to most graduate students than leftover salad,
and the FoodCam would be more useful if this information were included in emails.
